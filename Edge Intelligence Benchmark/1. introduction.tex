Edge computing emerges as a promising technical framework to overcome the challenges in cloud computing. In this AI era, AI applications are the most critical types of application on the edge. Edge Intelligence techniques are widely used to augment device, edge and cloud intelligence, and they are most demanding in terms of computing power, data storage, and network. 

However, Edge Intelligence is in the initial stage and does not have a comprehensive evaluation standard for scenarios with system on all three layers of Edge Intelligence. A comprehensive end-to-end Edge Intelligence benchmark suite is needed to measure and optimize the systems and applications.

\textbf{Contextual Metrics.} Typical application scenarios include smart city, smart building, smart home, smart factory, smart medical, autonomous vehicle, surveillance camera, and so on. These scenarios are complicated because of different kinds of client-side devices, a large quantity of heterogeneous data, privacy and security challenges. Most of these scenarios have a strict requirement for latency and network bandwidth. 

In the Edge Intelligence scenarios, the distribution of data and collaboration of varying application workloads are serious concerns not only for machine-learning algorithm accuracy, but also for response speed, networking, security, and privacy requirements. Because of the edge-cloud complexity, there is not yet a uniform application scenario to validate the architectures, systems, or specific algorithms in certain settings which will bring challenges to a general Edge Intelligence framework. For benchmarking, designing, and implementing edge computing systems or applications, we shall take a comprehensive view, considering varying applications. 

\textbf{End-to-end Test Case.} Compared with cloud-based framework, an edge-based framework adds a new layer, named the edge layer, on the basis of the traditional cloud-based framework. We then have the cloud, edge and device layers. In most existing Edge Intelligence framework, only the real-time data processing and model inference is conducted on the edge layer, while other complicated data processing is still executed on the cloud server.

On different layers, the collaboration of workloads and AI components are seriously complicated and a system point-of-view is of critical importance. For benchmarking, designing, and implementing Edge Intelligence systems or applications, we shall take an end-to-end view, considering all three layers and AI components. Unfortunately, the previous work, especially the previous benchmarking efforts ignore this point.

Moreover, Edge Intelligence is still in the initial stage with a lack of testbed. Because of the privacy issue, there is no incentive to share data, making it typically difficult for proposing a practical benchmark for individual developers and even companies (but things can be much different for open-source community).  

\textbf{Transparent Testbed.} Last but not the least, to demonstrate the state-of-the-art solutions and encourage collaborations and competitions, a leaderboard based on aforementioned metrics and test cases will be beneficial to both academic and industrial audience, so that they can share their latest progresses, post questions and answers,  and propose future directions.

To sum up, we believe that it is necessary to develop a benchmark suite including the above modules for Edge Intelligence, to reveal business requirements, best practice and final encourage facilitate cooperation between industry and acadamic community.
