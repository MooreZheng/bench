\subsection{Process}

\subsubsection{Preparation Phase} Among all procedures, the steps in which the preparation phase of the benchmarking process is highly concerned. More specifically, the determination of the comparison process in the preparation phase, which is the first stage of the benchmarking process, is of great importance in terms of the efficient use of the resources and their effects on the goals and success of the organization. An organization that wants to conduct benchmarking should firstly recognize its customers and identify factors that satisfy customers. The main objectives of the organization and the business processes supporting them are determined and the processes are determined in order of importance and the subject of benchmarking is determined.

The decision in which field to make a comparison-benchmarking is given by management as follows:
\begin{itemize}
    \item Customers are identified.
    \item Critical success factors (factors satisfying customers) are defined.
    \item Identify the key business processes that affect the organization's critical success factors.
    \item The processes that provide the most impact to the organization goals are determined and prioritized.
    \item Critical success factors and performance measures of these processes are reviewed and the desired processes for maintaining development are determined.
    \item It is decided whether to exchange information from the framework of information sharing rules with the possible benchmark partner related to these processes.
\end{itemize}


\subsection{Rules}
Edge-cloud Synergy AI Benchmark follows the PRDAERS benchmarking rules and methodology. The PRDAERS is the abbreviation of paper-and-pencil, relevant, diversity, abstractions, evaluation metrics and methodology, repeatable, and scaleable.

We first introduce the construction methodology of Edge-cloud Synergy AI Benchmark. Firstly, the typical scenarios are
extracted to present the numerous edge AI scenarios, and then the primary layers with micro, component and application benchmarks are extracted. The heterogeneous representative real-world
datasets are chosen as the dataset.

\textbf{Paper-and-pencil Approach}: Edge-cloud Synergy AI Benchmark can be specified only algorithmically in a paper-and-pencil approach. This benchmark specification is proposed firstly and reasonably divorced from individual implementations. In general, Edge-cloud Synergy AI Benchmark defines a problem domain in a high-level language.

\textbf{Relevant}: Edge-cloud Synergy AI Benchmark is domain-specific in edge computing and can be distinguished between different contexts. And in the other hand, Edge-cloud Synergy AI Benchmark is simplified and distillated of the real-world application. It abstracts the typical scenario and application.

\textbf{Diversity and Representatives}: Modern workloads show significant diversity in workload behavior with no single silverbullet application to optimize. Consequently, diverse workloads and datasets should be included to exhibit the range of behavior of the target applications. Edge-cloud Synergy AI Benchmark focuses on various edge AI problem domains and tasks.

\textbf{Abstractions}: Edge-cloud Synergy AI Benchmark contains different levels of abstractionsmicro, component, and end-to-end application benchmarks. For comprehensiveness and reality, Edge-cloud Synergy AI Benchmark models a real-world application, while for portability, the benchmark should be light-weight that can be portable across different systems and architectures. Thus, the benchmark should provide a framework that collectively runs as a end-to-end application and individually runs as a micro or component benchmark.

\textbf{Evaluation Metrics and Methodology}: The performance number of Edge-cloud Synergy AI Benchmark is simple, linear, orthogonal, and monotonic. Meanwhile, it is domain relevant. The metrics chosen by Edge-cloud Synergy AI Benchmark is relevant to edge-cloud synergy AI scenarios, such as accuracy, layency, and network metrics.

\textbf{Repeatable, Reliable, and Reproducible}: Since many egde computing deep learning workloads are intrinsically approximate and stochastic, allowing multiple different but equally valid solutions, it will raise serious challenges for Edge-cloud Synergy AI Benchmark.

\textbf{Scaleable}: Edge-cloud Synergy AI Benchmark should be scaleable. Thus, the benchmark users can scale up the problem size. 