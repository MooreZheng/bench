

\subsection{Environment}


\subsubsection{Object Detection}
\textbf{Helmet Detection:} 

\textbf{1. Architecture.} Helmet-Off Detection system contains the cameras and AI edge nodes connected to the edge-cloud, which do not require equipment rooms. There is at least one edge node for each of the five sites C1, C2, C3, S1, and S2. Cameras in each site are all connected to the edge stations via WiFi under a star network topology. Different from edge nodes and the edge cloud, cameras do not have computation power and are merely used to collect data. The simulation will be conducted according to the above setting to better simulate the real-world helmet-detection applications on the edge cloud.

\textbf{2. Dataset and Related Settings.} Among all the cameras, we take 49 cameras from all five sites as the training set. The rest 8 cameras are used as the test set, i.e., two cameras (S1-1, S1-2) from S1 and six cameras (S2-1, S2-2, S2-3, S2-4, S2-5, S2-6) from S2. That is, the ratio between the on-cloud training set and the on-edge test set is 9:1. 

\textbf{3. Contextual Metrics.} In the current helmet detection task, we focus on the detection capability of the helmet-off object. Thus, we measure the Recall, Precision, and F1-score to determine the Helmet-off detection capability of the model. 

\begin{equation} \label{equ:recall} 
Recall = \frac{TP}{TP + FN}, 
\end{equation}
\begin{equation} \label{equ:precision} 
Precision = \frac{TP}{TP + FP}, 
\end{equation}
\begin{equation} \label{equ:f1} 
F1-score = \frac{ 2 } {\frac{1}{Recall} + \frac{1}{Precision}},
\end{equation}
where $TP$, $FN$ and $FP$ denote the number of true positive, false negative and false positive results in the whole dataset. 

$Precision$ is implied as to the measure of the correctly identified positive cases from all the predicted positive cases. Thus, this metric is useful when the costs of \textit{False Positives} are high. $Recall$ is the measure of the correctly identified positive cases from all the actual positive cases. This metric is important when the cost of \textit{False Negatives} is high. $F1-score$ is the harmonic mean of Precision and Recall and gives a better measure of the incorrectly classified cases than the Accuracy Metric.

\textbf{4. Features.}

\textbf{5. Baselines and Hyperparameters.}

\textbf{6. Hardware.}

\textbf{7. Supported Learning Scheme.}


\subsubsection{Classification}




\textbf{Thermal Comfort Prediction:} 

\textbf{1. Architecture.}

\textbf{2. Dataset and Related Settings.}
ASHRAE Thermal Comfort Database (ATC II) is used to predict occupants' feeling of comfort, cold or hot. Data cleaning is conducted to remove samples with empty entries on significant feature items.

\textbf{3. Contextual Metrics.} 
To capture the predictive capabilities of our approach, we use the following metrics: Accuracy for evaluation.
\noindent
\begin{equation*}
\begin{split}
&ACC = TP / N; \\
\end{split}
\end{equation*}
where $TP$ denotes the number of true positive and $N$ denotes the number of testing data. 

\textbf{4. Features.} 
We categorize the data with the variables that can be used as raw features and the variables that can be used for prediction. For raw features, we have basic identifiers, e.g., age and sex. These are personalized data, yet it is easier to collect these data. There is information on the physical environment, which can be collected by building automation systems. In this paper, we conduct prediction on direct comfort metrics of thermal sensation and preference which are for the general thermal comfort process, while the unselected ones are for adaptive thermal comfort process and we leave the related investigation as future work. The important feature items include age, sex, taav, trav, velav, rh, et, set, tsens, pcc\_ag, day15\_ta, day06\_ta, and dayav\_et. 

\textbf{5. Baselines and Hyperparameters.}

\textbf{6. Hardware.}

\textbf{7. Supported Learning Scheme.}

 
\vspace{0.2cm} \noindent
\textbf{Defect Detection:}

\textbf{1. Architecture.}

One edge node connects to the cloud. 

\textbf{2. Dataset and Related Settings.}

In the evaluation, we chronologically order our three-month dataset and use a hold-out evaluation with a ratio of 10:1:1, i.e., the first 10-week data for training, 1-week data for evaluation and the remaining 1-week for testing.

\textbf{3. Contextual Metrics.}

To capture the predictive capabilities of our approach, we use the following metrics: Accuracy for evaluation.
\noindent
\begin{equation*}
\begin{split}
&ACC = TP / N; \\
\end{split}
\end{equation*}
where $TP$ denotes the number of true positive and $N$ denotes the number of testing data. 

\textbf{4. Features.}

\textbf{5. Baselines and Hyperparameters.}

\textbf{6. Hardware.}

All our experiments are conducted in a private cloud with 16 cores of 2.6GHz CPU and 64G memory.

\textbf{7. Supported Learning Scheme.}

1) Lifelong Learning. 2) Multi-task Learning. 


\subsubsection{Regression}

\textbf{Coke Quality Prediction:} 

\textbf{1. Architecture.}

\textbf{2. Dataset and Related Settings.}
Coke Production Dataset (CPD) is used to predict coke quality based on certain coal blending proportions. The dataset used in this study is collected from a factory for 6 months. With 108 samples of different coal blending combinations in total, the ratio between the training and testing is set as 8:2 and 22 samples are set as testing samples. In the training part for task forest, 25$\%$ of the training data are used for validation. 

\textbf{3. Contextual Metrics.} 
Predicting the continuous, scalar value of CSR is basically a regression problem, directly leading to our selection of the metrics of Mean Absolute Error (MAE) and Mean Squared Error (MSE). Also, the error rate metric is used to measure the relative bias of predictions. 
To avoid moderate predictions, matching the prediction vectors with the ground-truth vectors will end up with better predictions.
So to capture the trend tracking capability of the proposed approach, the Pearson Correlation Coefficient (PCCS), which is a common coefficient for measuring the covariance of two variables is used.

\begin{equation} \label{equ:mae} 
MAE = \frac{1}{J} \sum^J_{j=1} \frac{\sum_{i=1}^N |\hat{y}^{j}_{i} - y^{j}_{i}|}{N_j}, 
\end{equation}
\begin{equation} \label{equ:mse} 
MSE = \frac{1}{J} \sum^J_{j=1} \frac{\sum_{i=1}^N (\hat{y}^{j}_{i} - y^{j}_{i})^2}{N_j},
\end{equation}
\begin{equation} \label{equ:er} 
MAPE = \frac{1}{J} \sum^J_{j=1} \frac{\sum_{n=1}^{N_j} |\hat{y}^j_n - y^j_n|}{\sum_{n=1}^{N_j} y^j_n}, 
\end{equation}
\begin{equation} \label{equ:pccs} 
PCCS = \frac{1}{J} \sum^J_{j=1} \frac{\sum_{i=1}^{N_j} (\hat{y}^{j}_{i} - \hat{\mu}^j)(y^{j}_{i} - \mu^j)}{\sqrt{\sum_{i=1}^{N_j} (\hat{y}^{j}_{i} - \hat{\mu}^j)^2}\sqrt{\sum_{i=1}^{N_j}(y^{j}_{i} - \mu^j)^2}},
\end{equation}
where $J$ denotes the number of testing tasks; $N_j$ is the number of testing data on task $j$; $\hat{y}_n$ and $y_n$ are the estimation and the ground truth of the $n^\text{th}$ sample; $\hat{\mu}^j$ and $\mu^j$ are the mean of $\hat{y}_n$ and $y_n$, respectively.

\textbf{4. Features.}
To be more specific, for coal, data of ratios for different types of coals and properties for each type in the form of a 4-dimension vector showing values of Ash (Ad), Volatiles (Vdaf), Sulfur (St,d), and Caking Index (G) are available. By taking the ratio-weighted average value of each of the four properties, each coal blending sample is represented by a 4-dimension feature. For coke, data of Ad, Vdaf, Std, and G value for each sample, and the corresponding CSR value after coking the coals are available. The preprocessed 4-dimension feature for several coal blending combinations are used as input data, and the CSR value is used as corresponding labels. 

\textbf{5. Baselines and Hyperparameters.}

\textbf{6. Hardware.}

\textbf{7. Supported Learning Scheme.}







%1) average accuracy metric; 
%2) generalized - specifier real-world samples;
%3) optimization samples: delay, result (feasible; cost reduction);

\subsubsection{Re-identification}
\textbf{Person Re-identification} Person ReID aims to match the same person in non-overlapping camera views. 

(TODO: add metrics and more explanation on datasets)

\textbf{1. Architecture.}

\textbf{2. Dataset and Related Settings.}
We select 9 different datasets with significant variances in image amounts, identity numbers, scenes (indoor or outdoor), and the number of camera views. These properties lead to huge domain gaps among each other , simulating the statistical heterogeneity in reality. 

\textbf{3. Contextual Metrics.}

\textbf{4. Features.}

\textbf{5. Baselines and Hyperparameters.}

\textbf{6. Hardware.}

\textbf{7. Supported Learning Scheme.}