\subsection{Terminology}

\hspace{0.3cm}
\textbf{Benchmark} refers to the process of running a specific program or workload on a specific machine or system and measuring the resulting performance.

\textbf{Benchmark Suite} refers to a suite of benchmarks.

\textbf{AI} is short for Artifictial Intelligence.

\textbf{Edge-cloud Synergy AI Bench} refers to the Edge-cloud Synergy AI benchmark specification or implementation under different contexts. In some context, it refers to both Edge-cloud Synergy AI benchmark specification and implementation.

\textbf{Edge Computing} refers to a distributed computing paradigm which brings real-time computation and data closed to the user devices.

\textbf{Edge-cloud Synergy AI} refers to the AI scenarios under edge-cloud synergy implementation.

\textbf{Three Layer Framework} refers to the three layer framework in edge computing including the cloud server, the edge device, and the end devices.

\textbf{End-to-end Benchmark} refers to the end-to-end view which considers all three layers of the whole edge computing architecture. And it provides component benchmarks (such as training and inference) for users to decide which layer to execute the them for the better performance.

\textbf{Application Benchmark} refers to the benchmark that models the typical edge AI application .

\textbf{Component Benchmark} refers to the benchmark that models a component which can be deployed in each of three layers, such as training, inference and so on.

\textbf{Micro Benchmark} refers to single unit of computation designed for fine-grained evaluation and measure the performance of small piece of code.

\textbf{Reference Implementation} refers to the official benchmark implementation provided for reference.

\textbf{System under Test (SuT)} refers to the system that is being tested for correct operation.

