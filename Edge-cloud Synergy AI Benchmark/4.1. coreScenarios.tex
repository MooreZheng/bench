\subsection{Core Scenarios}

\subsubsection{Object Detection}
\textbf{Helmet Detection.} Although industrial safety helmets are known for preventing head injuries, in many countries, the use of helmets is low due to the lack of police power to enforcing helmet laws. Helmet detection, i.e., detecting whether an individual is wearing a helmet or not, is thus needed in practice. 

Based on cloud computing together with the recent development of edge computing approaches, recent studies on automatic object detection also tend to apply computer vision and image processing methods on not only the cloud but also edge devices, e.g., smart cameras and edge servers. 


\subsubsection{Classification}
\textbf{Thermal Comfort Prediction.} Thermal comfort is defined as the condition of mind that expresses satisfaction with the thermal environment, or simply put, it shows whether an occupant feels cold or hot. Thermal comfort provides a quantitative assessment that links the setting of indoor thermal environment parameters to the subjective evaluation of the occupants. It has been widely applied for choosing building design alternatives, and for operating the set-point management of the air-conditioning system. It also heavily influences energy consumption levels in buildings. In fact, building operators are reluctant to adopt new energy conservation technologies without having a clear knowledge of their impact on occupant comfort.

Recently, with advances in smart devices and sensing technologies, developing data-driven approaches to achieve thermal comfort has become possible and is highly advocated. While the accuracy has been continuously improving, many of these studies require sufficient local samples collected before learning.

\vspace{0.2cm} \noindent
\textbf{Defect Detection.} In recent years, the manufacturing process is moving towards a higher degree of automation and improved manufacturing efficiency. During this development, smart manufacturing increasingly employs computing technologies, for example, with a higher degree of automation, there is also a higher risk in product defects; thus, a number of machine learning models have been developed to detect defectives in the manufacturing process.  

Defects are an unwanted thing in manufacturing industry. There are many types of defect in manufacturing like blow holes, pinholes, burr, shrinkage defects, mould material defects, pouring metal defects, metallurgical defects, etc. For removing this defective product all industry have their defect detection department. But the main problem is this inspection process is carried out manually. It is a very time-consuming process and due to human accuracy, this is not 100\% accurate. This can because of the rejection of the whole order. So it creates a big loss in the company.


\subsubsection{Regression}
\textbf{Coke Quality Prediction.} Coke quality prediction is to predict coke quality based on certain coal blending proportions. The coke quality is measured by one scalar property, CSR (coal strength after reaction), for which higher values indicate higher quality. Some properties of different types of coals will be tested before being sent to the coke oven, and then after a series of physical and chemical reactions within high-temperature dry distillation, coke will be extracted from the ovens around 24 hours later. Several properties of coal and coke are measured before and after the blending process. 



\subsubsection{Re-identification}
\textbf{Person Re-identification} Given a query image, the person ReID system aims to retrieve images with the same identity from disjoint cameras, based on their similarities. It has wide applications such as video surveillance and content-based video retrieval. 

(TODO: add more explanation of the scenario)