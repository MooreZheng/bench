\subsection{Scheme}

Directories

- benchmark

- algorithm

- scenarios

\subsubsection{Lifelong Learning}

\textbf{1. AI-based Testing}

Baseline testing procedures. 

\textbf{1.1. Non-lifelong learning baseline.}

(1) Predicted Mean Vote (PMV)~\cite{iso05}, which is a well-known synthetic model built with domain knowledge instead of machine learning; 

(2) Single Task Learning (STL)~\cite{peng17data}, which learns a single model by pooling together initial training data from all tasks; 

(3)
Distinct Multi-task Learning (DMTL)~\cite{yang18}, which conducts metadata-assisted multi-task prediction on thermal comfort given inital training data. It has leverage metadata for task allocation, but without knowledge maintainance for continual lifelong service. 

\textbf{1.2. Lifelong learning baseline.}

(4) Online single task learning (OSTL)~\cite{anderson08theory}, e.g., STL with data come continually. 

(5) Traditional lifelong learning (TLL)~\cite{silver13lifelong} which learns online tasks models under given tasks, e.g., DMTL with data arrive continually. 

(6) Lifelong memory learning (LML)~\cite{altman92introduction}, which learns online tasks models with knowledge as samples instead of meta knowledge.



\textbf{2. AI-System-based Testing}



\subsection{Scenario}

\subsubsection{Person Reidentification}