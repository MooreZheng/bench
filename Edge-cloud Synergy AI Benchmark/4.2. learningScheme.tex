\subsection{Learning Scheme}


\begin{table}[h]
\caption{Common Metrics.} 
\label{tab:common}
\begin{tabular}{|l|l|p{3cm}|p{4cm}|p{4cm}|}
\hline
Metric                 & Type      & Computation & Description                                                                & Reference \\ \hline
End-to-end Delay
& AI System 
& 
\parbox{3cm}{
    \begin{equation} \label{equ:ed} 
    T_{start} - T_{end}
    \end{equation}
}
& Time needed by data preprocess, feature enginneering, model inference etc. 
&           
\\ \hline

\end{tabular}
\end{table}


\noindent \textbf{Coke Quality Prediction}
\begin{itemize}
    \item Objective: - 
    \item Constaint: Metric \ref{equ:ed}
        \begin{itemize} 
        \item 6ms per sample 
        \item 100 samples in one batch, i.e., 0.6s per batch
        \end{itemize}
\end{itemize}

\noindent \textbf{Defect Detection}
\begin{itemize}
    \item Objective: - 
    \item Constaint: Metric \ref{equ:ed}
        \begin{itemize} 
        \item 500 fps, i.e., 500 images per second 
        \end{itemize}
\end{itemize}

\subsubsection{Federated Learning}

The metrics are shown in Table~\ref{tab:fl}.

\noindent \textbf{Thermal Comfort Prediction}
\begin{itemize}
    \item Objective: Metric \ref{equ:aoaoc}, \ref{equ:voaoc}, 
    \item Constaint: Metric \ref{equ:cv}
\end{itemize}

\noindent \textbf{Coke Quality Prediction}
\begin{itemize}
    \item Objective: Metric \ref{equ:aoaoc}, \ref{equ:voaoc}
    \item Constaint: Metric \ref{equ:cv}
\end{itemize}

\begin{table}[h]
\caption{Federated Learning Metrics.} 
\label{tab:fl}
\begin{tabular}{|p{3cm}|l|p{6cm}|p{3cm}|p{1.5cm}|}

\hline
Metric                 & Type      & Computation & Description                                                                & Reference \\ \hline
Accuracy on client
& AI 
& 
\parbox{3cm}{
    \begin{equation} \label{equ:aoc} 
    \begin{split}
& acc_{\text{client}} = (TP_{\text{client}} 
 / (P_{\text{client}} + N_{\text{client}})
    \end{split} 
    \end{equation} 
}
& accuracy on one client 
& \cite{DBLP:conf/iclr/LiSBS20}       
\\ \hline

Average of accuracy on clients
& AI
&
\parbox{3cm}{
    \begin{equation} \label{equ:aoaoc} 
    Avg(acc_{\text{client}}) = \sum acc_{\text{client}} / N
    \end{equation}
}
& average accuracy on one client 
& \cite{DBLP:conf/iclr/LiSBS20}       
\\ \hline

Variance of accuracy on clients
& AI
&
\parbox{3cm}{
    \begin{equation} \label{equ:voaoc} 
    \begin{split}
    Var(acc_{\text{client}}) =
    & \sum (acc_{\text{client}}^2 - \\ 
    & avg(acc_{\text{client}})) / N
    \end{split}
    \end{equation}
}
& Variance accuracy on one client 
& \cite{DBLP:conf/iclr/LiSBS20}       
\\ \hline

Accuracy on server
& AI
&
\parbox{3cm}{
    \begin{equation} \label{equ:aos} 
    \begin{split}
acc_{\text{server}} = 
& (TP_{\text{server}} + TN_{\text{server}}) / \\
& (P_{\text{server}} + N_{\text{server}} )
    \end{split}
    \end{equation}
}
& 
Accuracy on server
& \cite{DBLP:conf/aistats/McMahanMRHA17}
\\ \hline

Communication volume
& AI System
&
\parbox{3cm}{
    \begin{equation} \label{equ:cv}  
    \begin{split}
& vol_{\text{comm}} = 
 \sum_{\text{communication round}} 
\text{uploaded} \\
& \text{resource size}
    \end{split}
    \end{equation}
}
& 
Communication volume
& \cite{DBLP:conf/aistats/McMahanMRHA17}
\\ \hline

\end{tabular}
\end{table}

\begin{table}[h!]
\caption{Lifelong Learning Metrics.} 
\label{tab:ll}
\begin{tabular}{|p{2cm}|l|p{6cm}|p{3cm}|p{1.5cm}|}

\hline
Metric                 & Type      & Computation & Description                                                                & Reference 
\\ \hline

Average of Multi-task Accuracy
& AI 
& 
\parbox{3cm}{
    \begin{equation} \label{equ:aoma} 
    \begin{split}
\text{Avg(Acc}_\text{task}) = \sum^N_{i \geq j} R_{i,j} / N
    \end{split} 
    \end{equation} 
}
& 
& \cite{DBLP:journals/corr/abs-1810-13166,DBLP:conf/nips/Lopez-PazR17}       
\\ \hline

Average of Multi-task Backward Transfer
& AI 
& 
\parbox{3cm}{
    \begin{equation} \label{equ:aombt} 
    \begin{split}
& \text{Avg(BWT}_\text{task}) \\
& = \sum^{N}_{i=2}\sum^{i-1}_{j=1} (R_{i,j}-R_{j,j}) / N
    \end{split} 
    \end{equation} 
}
&  
& \cite{DBLP:journals/corr/abs-1810-13166,DBLP:conf/nips/Lopez-PazR17}       
\\ \hline


Standard deviation of Multi-task Backward Transfer
& AI 
& 
\parbox{3cm}{
    \begin{equation} \label{equ:sdombt} 
    \begin{split}
& \text{Std(BWT}_\text{task}) = \sigma(R_{i,j}-R_{j,j}), \\ 
& \forall i \in [2, N], j \in [1, i-1]
    \end{split} 
    \end{equation} 
}
&  
& \cite{DBLP:journals/corr/abs-1810-13166,DBLP:conf/nips/Lopez-PazR17}       
\\ \hline


Model Size Efficiency
& AI System
& 
\parbox{3cm}{
    \begin{equation} \label{equ:mseff} 
    \begin{split}
& MSEff = min(1, \sum_{i=1}^N Mem(\theta_1) \\
& / Mem(\theta_i) / N)
    \end{split} 
    \end{equation} 
}
& The memory size of model hi quantified in terms of parameters at each task i, Mem(i), should not grow too rapidly with respect to the size of the model that learned the first task, Mem(1). 
& \cite{DBLP:journals/corr/abs-1810-13166,DBLP:conf/nips/Lopez-PazR17}       
\\ \hline

Model Size Efficiency
& AI System
& 
\parbox{3cm}{
    \begin{equation} \label{equ:ssse} 
    \begin{split}
& SSSEff = 1 - min(1, 1/N * \sum_{i=1}^N \\
& Mem(M_i) / Mem(D))
    \end{split} 
    \end{equation} 
}
&  
& \cite{DBLP:journals/corr/abs-1810-13166,DBLP:conf/nips/Lopez-PazR17}       
\\ \hline

\end{tabular}
\end{table}




\subsubsection{Lifelong Learning}

The metrics are shown in Table~\ref{tab:ll}. Given the train-test accuracy matrix $R \in \mathbb{R}^{N \times N}$, which contains in each entry $R_{i,j}$ the test classification accuracy of the model on task $t_j$ after observing the last sample from task $t_i$, Accuracy (A) considers the average accuracy for training set $Tr_i$ and test set $Te_j$ by considering the diagonal elements of $R$, as well as all elements below it. 

\vspace{0.2cm}
\noindent \textbf{Defect Detection}
\begin{itemize}
    \item Objective: Metric \ref{equ:aoma} \ref{equ:aombt} \ref{equ:aombt} 
    \item Constaint: Metric \ref{equ:mseff}, \ref{equ:ssse}  
\end{itemize}


%\textbf{Layer Metric.} Default: Training on cloud; inference on edge. To be developed. 

%\textbf{Module Metric.} Default: End-to-end. To be developed.  

%\textbf{AI System Metric} Default: algorithm. Not only algorithm, but also optimization, decision making, etc. 

%\textbf{Setting.} Fixed test dataset. Train: validate: Test:

%1) For objective: (train + valid) (1+1 - 10+1), fixed test set 1=(0.2, 0.2, 0.2, 0.2, 0.2), return worst result; 

%2) For constraint: given test samples and threshold (coke $\geq$ point-wise 95\%), check the number of test samples that pass the threshold (UR) and averge inference time;








\subsubsection{Edge-cloud Synergy Inference}

The metrics are shown in Table~\ref{tab:esi}.

\begin{table}[h!]
\caption{Edge-cloud Synergy Inference Metrics.} 
\label{tab:esi}
\begin{tabular}{|p{2cm}|l|p{6cm}|p{3cm}|p{1.5cm}|}

\hline
Metric                 & Type      & Computation & Description                                                                & Reference 
\\ \hline

Compression ratio
& AI System
& 
\parbox{3cm}{
    \begin{equation} \label{equ:cr}
    \begin{split}
cr =  \text{edge model size}  /  \text{cloud model size}
    \end{split} 
    \end{equation} 
}
& 
&      
\\ \hline

Upper cloud ratio
& AI System
& 
\parbox{3cm}{
    \begin{equation} \label{equ:ucr}
    \begin{split} 
& ucr = \text{upload cloud data volume} /  \\
& \text{all data volume} 
    \end{split} 
    \end{equation} 
}
& 
&      
\\ \hline

Calculation amount
& AI System
& 
\parbox{3cm}{
    \begin{equation} \label{equ:flops} 
    \begin{split}
FLOPs
    \end{split} 
    \end{equation} 
}
& 
&      
\\ \hline

Throughput
& AI System
& 
\parbox{3cm}{
    \begin{equation} \label{equ:fps} 
    \begin{split}
FPS =  \text{Frames per second} 
    \end{split} 
    \end{equation} 
}
& 
&      
\\ \hline

Model size
& AI System
& 
\parbox{3cm}{
    \begin{equation} \label{equ:mz} 
    \begin{split}
Model size (MB)
    \end{split} 
    \end{equation} 
}
& 
&      
\\ \hline

Mean average precision
& AI 
& 
\parbox{3cm}{
    \begin{equation} \label{equ:map} 
    \begin{split}
mAP
    \end{split} 
    \end{equation} 
}
& 
&      
\\ \hline
\end{tabular}
\end{table}


